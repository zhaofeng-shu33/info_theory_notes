% Homework template for Inference and Information
% UPDATE: September 26, 2017 by Xiangxiang
\documentclass[a4paper]{article}
\usepackage{ctex}
\ctexset{
proofname = \heiti{证明}
}
\usepackage{amsmath, amssymb, amsthm}
% amsmath: equation*, amssymb: mathbb, amsthm: proof
\usepackage{moreenum}
\usepackage{mathtools}
\usepackage{url}
\usepackage{bm}
\usepackage{enumitem}
\usepackage{graphicx}
\usepackage{subcaption}
\usepackage{booktabs} % toprule
\usepackage[mathcal]{eucal}
\usepackage[thehwcnt = 4、5]{iidef}
\usepackage{makecell}
\DeclarePairedDelimiter\abs{\lvert}{\rvert}
\thecourseinstitute{清华大学深圳研究生院}
\thecoursename{应用信息论}
\theterm{2018年春季学期}
\hwname{习题}
\slname{\heiti{解}}
\begin{document}
\courseheader
\name{赵丰}


\begin{enumerate}[label = \arabic*]
  \setlength{\itemsep}{3\parskip}
\item 设有离散无记忆信道,设$\bar{I}(X,Y)$ 是输入输出的函数,为一随机变量,取值为
$\bar{I}(a_i, b_j) = \log { P(X = a_i, Y = b_j) \over P(X=a_i) P(Y=b_j) }$,取该值的概率为$P(X = a_i, Y=b_j)$。
易知 $X$与$Y$的互信息为 $\bar{I}(X,Y)$的均值。证明当$\Var[\bar{I}(X, Y)] = 0$时, $\E[ \bar{I}(X; Y) ] $ 达到信道
容量。
\begin{solution}
$ \Var[\bar{I}(X, Y)] = 0 \Rightarrow \bar{I}(X,Y)  = C' $ 因此 $I(X=a_i; Y) = \sum_{j} P(Y=b_j | X = a_i) \bar{I}(a_i, b_j) = C' $
由离散无记忆信道的信道容量定理可得平均互信息达到信道容量。 
\end{solution}
  \item 设某信道的输入$X$ 取值为 $\{ +1, -1 \}$, 又信道有加性噪声$n$, 其分布密度为 $ p(n) = \begin{cases}
  { 1 \over 4} , & \abs{n} \leq 2 \\ 0, & \abs{n} > 2 \end{cases}$, 求信道容量。
  \begin{solution}
  \begin{align*}
  I(X;Y) & = H(Y) - H(Y|X) \\
  & = H(Y) - 2
 \end{align*}
 当 $X\sim Bern({1 \over 2})$ 时,
 $ p(y) =  \begin{cases} { 1\over 8}, & 1 \leq \abs{y}\leq 3 \\
 { 1 \over 4}, & \abs{y} < 1 \\
 0, & \abs{y} > 3 
 \end{cases} $
 在这种情形下 $ H(Y) = 2.5 \Rightarrow I(X; Y) = 0.5$

 对于 $\Pr( X = 1 ) = p $ 的情形, $ p(y) = \begin{cases} { 1 \over 4 } p, & 1 \leq y \leq 3 \\ {1 \over 4} (1-p), & -3 \leq y \leq -1 
 \end{cases} $, 其余与 $ p = \frac{1}{2} $ 相同。 则
 $ I(X;Y) = {1 \over 2} h(p) $, 其中 $h(p)$ 为二元熵函数。因此当 $ p = { 1 \over 2} $时$I(X;Y)$最大, 达到信道容量 $ 
 C = {1 \over 2} $。
 \end{solution}
 
\item 设$X$ 和 $Y$ 为信道的输入和输出, 两者均取值于集合 $ A = \{ a_1, a_2, \dots, a_K\}$。
已知 $ p ( x = a_k ) = p_k, p( y = a_j | x = a_k ) = p_{kj} $, 定义 $ P_e = \displaystyle\sum_{k} p_k \sum_{j \neq k} p_{kj} $
求证:
\begin{equation}\label{eq:HXY}
H(X|Y) \leq P_e \log (K - 1) + H(P_e)
\end{equation}
其中 $ H(P_e) $ 是关于 $P_e$ 的二元熵函数。
\begin{proof}
设 $p( y = a_j | x = a_k )$ 描述了一个离散无记忆的信道, $X$ 是信道输入, $Y$ 是信道输出, 现设 $\hat{X} = Y$, 即用信道输出值来解码 $X$, 错误概率为 $P_e$, 由 Fano 不等式可知 ~\eqref{eq:HXY} 成立。
\end{proof}
\item 已知信道转移概率矩阵如表~\ref{tab:transfer_matrix} 所示, 求此信道的信道容量。
\begin{table}[!ht]
\centering
\begin{tabular}{c|cccc}
\diaghead(-1, 1){xx}{x}{y} & 0 & 1 & 2 & 3 \\
\hline
0 & 1/3 & 1/3 & 1/6 & 1/6 \\
1 & 1/6 & 1/3 & 1/6 & 1/3 \\
\end{tabular}
\caption{信道转移概率矩阵}\label{tab:transfer_matrix}
\end{table}
\begin{solution}
由准对称信道的信道容量定理, 当输入分布为 $Bern( { 1 \over 2 } ) $ 时,达到信道容量。
可以求出此时 $ C = I(X; Y) = H(Y) - H(Y|X) = 0.041 $
\end{solution}
\item 设有信道,输入 $X$ 的字母表为: $\{0, 1, 2, \dots, K-1\}$, 噪声为独立加性噪声 $Z$, $Z$ 的取值
也在 $\{0, 1, 2, \dots, K-1\}$ 的集合中, 但两者相加为模 $K$ 相加, 即 输出 $ Y = X \bigoplus Z $(模 $K$),试求
此信道的信道容量。
\begin{solution}
此 DMC 信道是对称信道,转移概率矩阵是 Toeplitz 矩阵。行元素为 $P_Z(0), P_Z(1), \dots P_Z(K-1) $, 由DMC对称信道的
信道容量公式得 $ C = \log (K) - H(Z) $
\end{solution}
\item 设有输入为 $X$ 输出为 $Y = [Y_1, Y_2 ] $ 的高斯信道, 其中 $Y_1 = X + Z_1, Y_2 = X + Z_2 $, $X$ 的最大功率受限 为 $P$, $(Z_1,Z_2) \sim N_2(0, K)$, 其中 $ K = \begin{bmatrix} \sigma^2 & \rho \sigma^2 \\ \rho \sigma^2 & \sigma^2 \end{bmatrix} $,试求:
\begin{enumerate}[label=\arabic*)]
\item $ I(X;Y_1, Y_2) = I(X; Y_1) + I(X; Y_2) - I(Y_1; Y_2) + I(Y_1; Y_2 | X) $
\item $ \rho = 1 $ 时的信道容量。
\end{enumerate}
\begin{solution}
\begin{enumerate}[label=\arabic*)]
\item
\begin{align*}
 I(X; Y_1, Y_2 ) & = H(X) + H(Y_1, Y_2) - H(X, Y_1, Y_2) \\
 I(X; Y_1) & = H(X) + H(Y_1) - H(X, Y_1) \\
 I(X; Y_2) & = H(X) + H(Y_2) - H(X, Y_2) \\
 I(Y_1; Y_2) & = H(Y_1) + H(Y_2) - H(Y_1, Y_2) \\
 I(Y_1; Y_2 | X ) & = H(Y_1 | X) + H(Y_2 | X) - H(Y_1, Y_2 | X) \\
 & = H(X, Y_1) + H(X, Y_2) - H(X, Y_1, Y_2) - H(X)
\end{align*}
直接计算有 $ I(X;Y_1, Y_2) = I(X; Y_1) + I(X; Y_2) - I(Y_1; Y_2) + I(Y_1; Y_2 | X) $。
\item $ \rho = 1 $ 时的信道容量。
$ \rho = 1 $时 $Z_1$ 与 $Z_2 $ 几乎处处相等 $ \Rightarrow Y_1 \overset{as}{=} Y_2$。
由 1) 可以得到 $ I(X; Y_1, Y_2) = I(X; Y_1) $ 由已知结论, $ C = { 1\over 2} \log (1+ { P \over \sigma^2})$
\end{enumerate}
\end{solution}
\end{enumerate}
\end{document}


%%% Local Variables:
%%% mode: late\rvx
%%% TeX-master: t
%%% End:
