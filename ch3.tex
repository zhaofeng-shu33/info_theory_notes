\documentclass{article}
\usepackage{ctex}
\usepackage{bm}
\usepackage{enumitem}
\usepackage{amsmath,amsthm,amsfonts,amssymb}
\newtheorem{definition}{定义}
\newtheorem{thm}{定理}
\newtheorem{remark}{注}
\newtheorem{eg}{例}
\newtheorem{pro}{命题}
\newtheorem{cor}{推论}
\def\P{\textbf{P}}
\def\E{\mathbb{E}}
\usepackage{mathtools}
\DeclarePairedDelimiter\ceil{\lceil}{\rceil}
\begin{document}
\begin{enumerate}
\item 熵率

\begin{definition}
随机序列$X_n$的熵率定义为
\begin{equation}
H_{\infty}(X)=\lim_{n\to \infty} \frac{1}{n} H(X_1,\dots,X_n)
\end{equation}
\end{definition}
\begin{thm}
对于离散平稳信源,若$H(X_1)<\infty$,则$H_{\infty}(X)$存在且
\begin{equation}
H_{\infty}(X)=\lim_{n\to \infty} H(X_n | X_{n-1},X_{n-2},\dots,X_1)
\end{equation}
\end{thm}
\begin{proof}
\begin{align*}
H(X_n | X_{n-1},X_{n-2},\dots,X_1) \leq  & H(X_n | X_{n-1},X_{n-2},\dots,X_2) \\
=& H(X_{n-1} | X_{n-2},X_{n-3},\dots,X_1)
\end{align*}
所以 $H(X_n | X_{n-1},X_{n-2},\dots,X_1)$单调递减。
\begin{align*}
\Rightarrow & \frac{1}{n} H(X_1,\dots,X_n)=\frac{1}{n}\sum_{i=1}^n H(X_i|X_{i-1},X_1) \\
\Rightarrow & \frac{1}{n} H(X_1,\dots,X_n) \to \lim_{n\to \infty}H(X_n | X_{n-1},X_{n-2},\dots,X_1) \text{ as } n\to \infty
\end{align*}
\end{proof}
以下分别针对三种常见的情形给出熵率的计算公式:
\begin{enumerate}[label=(\alph*)]
\item 独立同分布:$H(X_1,\dots,X_n)=\sum_{i=1}^n H(X_i)=nH(X_1) \Rightarrow H_{\infty}(X)=H(X_1)$
即在独立同分布的情况下熵率等于熵。
\item 平稳的马氏链:$H(X_n | X_{n-1},X_{n-2},\dots,X_1)=H(X_n|X_{n-1})=H(X_2|X_1)$
设平稳分布为$\pi$,$\pi_i$表示处于状态$i$的概率。$p_{ij}=P(X_2=j|X_1=i)$,则
\begin{align*}
H(X_2|X_1) = & \sum_{i\in \mathcal{X}} \pi_i H(X_2|X_1=i)\\
= & -\sum_{i,j\in \mathcal{X}} \pi_i p_{ij}\log p_{ij}
\end{align*}

\begin{eg}
考虑一个两状态的马氏链,转移概率矩阵为
$
\begin{bmatrix}
1-\alpha & \alpha \\
\beta & 1-\beta
\end{bmatrix}
$
解下面的方程
$$
[\pi_1,\pi_2]=[\pi_1,\pi_2]\begin{bmatrix}
1-\alpha & \alpha \\
\beta & 1-\beta
\end{bmatrix}
$$
得
\begin{align*}
\pi_1 = & \frac{\beta}{\alpha+\beta} \\
\pi_2 = & \frac{\alpha}{\alpha+\beta}
\end{align*}
假设两状态分别为1和2,则$H(X_2|X_1=1)=-(1-\alpha)\log(1-\alpha)-\alpha\log\alpha=h(\alpha)$
同理$H(X_2|X_1=2)=h(\beta)$,因此对两状态的马氏链,熵率为$\pi_1 h(\alpha)+\pi_2 h(\beta)$
\end{eg}
\item 隐马尔科夫模型
\begin{thm}
$X_i$是平稳马氏链,$Y_i=f(X_i)$,则
\begin{align}
H(Y_n|Y_{n-1},\dots,Y_1,X_1) \leq & H_{\infty}(Y) \leq H(Y_n|Y_{n-1},\dots,Y_1) \label{eq:ineqH}\\
\lim_{n\to \infty} H(Y_n | Y_{n-1},\dots,Y_1,X_1) = & H_{\infty}(Y) = \lim_{n\to\infty} H(Y_n|Y_{n-1},\dots,Y_1) 
\end{align}
\end{thm}
\begin{proof}
上两式右端由平稳性可得,对于左端,首先说明$H(Y_n|Y_{n-1},\dots,Y_1,X_1)$ 是$n$的增函数,
这是因为
\begin{align*}
H(Y_n|Y_{n-1},\dots,Y_1,X_1) = & H(Y_n | Y_{n-1},\dots,Y_1,X_1,X_0) \\
= & H(Y_n | Y_{n-1},\dots,Y_1,X_1,X_0,Y_0)\\
\leq & H(Y_n | Y_{n-1},\dots,Y_1,Y_0,X_0) \\
= & H(Y_{n+1}|Y_n ,\dots, Y_2,Y_1,X_1)
\end{align*}
所以$\lim_{n\to \infty}H(Y_n|Y_{n-1},\dots,Y_1,X_1)$存在。
下面证明两极限相等,即证明当$n\to\infty$时,$H(Y_n|Y_{n-1},\dots,Y_1)-H(Y_n|Y_{n-1},\dots,Y_1,X_1)=I(X_1;Y_n|Y_{n-1},\dots,Y_1)\to 0$
把$I(X_1;Y_n|Y_{n-1},\dots,Y_1)$看成某个级数的通项:
$$
\sum_{i=1}^n I(X_1;Y_i|Y_{i-1},\dots,Y_1) = I(X_1;Y_1,\dots,Y_n)\leq H(X_1)
$$   
所以级数 $\sum_{i=1}^{\infty} I(X_1;Y_i|Y_{i-1},\dots,Y_1)$收敛$\Rightarrow I(X_1;Y_n|Y_{n-1},\dots,Y_1)\to 0$
即
$$
\lim_{n\to\infty}H(Y_n|Y_{n-1},\dots,Y_1,X_1)=\lim_{n\to\infty}H(Y_n|Y_{n-1},\dots,Y_1)=H_{\infty}(Y)
$$
并由$H(Y_n|Y_{n-1},\dots,Y_1,X_1)$递增的特性知\eqref{eq:ineqH}式左端成立。
\end{proof}
\end{enumerate}
\item 典型集
\begin{definition}
设$X_1,\dots,X_n\sim p(x)$,i.i.d.,则关于$p(x)$的典型集定义为以下序列的集合:
\begin{equation}
A^{(n)}_{\epsilon} = \{ (x_1,\dots,x_n) \in \mathcal{X}^n: 2^{-n[H(x)+\epsilon]} \leq p(x_1,\dots,x_n) \leq 2^{-n[H(x)-\epsilon]}  \}
\end{equation}
\end{definition}
典型集具有以下性质:
\begin{enumerate}[label=(\alph*)]
\item 若$(x_1,x_2,\dots,x_n) \in A^{(n)}_{\epsilon}$,则
\begin{equation}
H(X) - \epsilon \leq -\frac{1}{n} \log p(x_1,\dots,x_n) \leq H(X) + \epsilon
\end{equation}
\item 任意固定$\epsilon >0 $,当$n$充分大时,
$\Pr\{A^{(n)}_{\epsilon}\} \geq 1-\epsilon$
\begin{proof}[证明]
由性质(a),
$$
\Pr\{A^{(n)}_{\epsilon}\} = \Pr\{|-\frac{1}{n} \log p(X_1,\dots,X_n)-H(X)|<\epsilon \} 
$$
由弱大数定律得:
$$
-\frac{1}{n} \log p(X_1,\dots,X_n) = \frac{1}{n}\sum_{i=1}^n [-\log p(X_i)] \xrightarrow{P} \E_X[-\log p(X)]=H(X)
$$
根据依概率收敛的定义得证。
\end{proof}
\item $|\cdot|$表示集合的元素个数,则
$|A_{\epsilon}^{(n)}|\leq 2^{n[H(X)+\epsilon]}$
\begin{proof}[证明]
\begin{align*}
|A_{\epsilon}^{(n)}| & = \sum_{\bm{x}\in A_{\epsilon}^{(n)}} 1 \\
 & \leq \sum_{\bm{x}\in A_{\epsilon}^{(n)}} p(x_1,\dots,x_n) 2^{n[H(X)+\epsilon]} \\
 & \leq 2^{n[H(X)+\epsilon]} \sum_{\bm{x}\in \mathcal{X}^n} p(x_1,\dots,x_n) \\
 & = 2^{n[H(X)+\epsilon]}
\end{align*}
\end{proof}
\item $\forall \epsilon >0, n$充分大时,
$$
A_{\epsilon}^{(n)} \geq (1-\epsilon)2^{n[H(X)-\epsilon]}
$$
\begin{proof}[证明]
由(b)已知
$$
\Pr\{A^{(n)}_{\epsilon}\} = \sum_{x\in A^{(n)}_{\epsilon}} p(x_1,\dots,x_n) \geq 1-\epsilon
$$
\begin{align*}
|A_{\epsilon}^{(n)}| & = \sum_{\bm{x}\in A_{\epsilon}^{(n)}} 1 \\
 & \geq \sum_{\bm{x}\in A_{\epsilon}^{(n)}} p(x_1,\dots,x_n) 2^{n[H(X)-\epsilon]} \\
 & \geq (1-\epsilon)2^{n[H(X)-\epsilon]} 
\end{align*}
\end{proof}
\end{enumerate}
\begin{definition}
如果$\forall \delta>0$,当$n$充分大时,$\Pr\{B^{(n)}_{\delta}\}\geq 1-\delta$,
则称$B_{\delta}^{(n)}\subset \mathcal{X}^n$为包含大多数概率的子集(高概率集)。
\end{definition}
下面的定理说明$A_{\epsilon}^{(n)}$ 在一阶指数意义下是最小的高概率集:
\begin{thm}
设$X_1,\dots,X_n$ i.i.d. $\sim p(x)$,固定$\delta <\frac{1}{2}$ 及$B_{\delta}^{(n)}$,则对$\forall \delta'>0$,当$n$充分大时,
\begin{equation}
\frac{1}{n}\log |B_{\delta}^{(n)}| \geq H(X_1)-\delta'
\end{equation}
\end{thm}
\begin{proof}[证明]
由$P(A\cap B)>1-P(\bar{A})-P(\bar{B})$得到当$n$充分大时,
$$
P(B_{\delta}^{(n)} \cap A_{\epsilon}^{(n)}) >1-\epsilon-\delta
$$
\begin{align*}
|B_{\delta}^{(n)}| & \geq |B_{\delta}^{(n)} \cap A_{\epsilon}^{(n)}| \\
& = \sum_{\bm{x}\in B_{\delta}^{(n)} \cap A_{\epsilon}^{(n)}} 1 \\
& \geq \sum_{\bm{x}\in B_{\delta}^{(n)} \cap A_{\epsilon}^{(n)}}p(x_1,\dots,x_n)2^{n[H(x)-\epsilon]}\\
& \geq (1-\epsilon-\delta)2^{n[H(x)-\epsilon]}
\end{align*}
\end{proof}
\item 变长编码

$x\in \mathcal{X},C(x)$表示对应$x$的码字,$l(x)$表示$C(x)$的长度。若信源编码$C$中无任何码字是其他的前缘,则称$C$为即时码。
\begin{thm}[Kraft 不等式]\label{thm:Kraft}
$D$元字母表上的即时码,设有$m$个码字,码长分别为$l_1,\dots,l_m$,则有:
\begin{equation}
\sum_{i=1}^m D^{-l_i} \leq 1
\end{equation}
\end{thm}
\begin{proof}[证明]
设$y=(y_1,\dots,y_{l_i})$为码字,$y\leftrightarrow D$元小数 $0.y_1y_2\dots y_{l_i}\triangleq \sum_{j=1}^{l_i}y_j D^{-j}$
$\leftrightarrow$ 小区间 $I_{y}=(0.y_1y_2\dots y_{l_i},0.y_1y_2\dots y_{l_i}+D^{-l_i})$。
注意到小区间的右端点是在$D$位小数的末位加1,如果对于另外一个$D$位小数$\tilde{y}$对应的小区间与$y$对应的小区间相交,不妨设$0.\tilde{y_1}\dots\tilde{y_{l_k}}\in I_{y}$,
则可以说明$y$是$\tilde{y}$码字的前缘,这与即时码的定义相矛盾。因此各码字对应的小区间互不相交,其区间总长度为$\sum_{i=1}^m D^{-l_i}$小于$[0,1]$区间的长度1。

\end{proof}
\begin{thm}
任给即时码$C$,$L(C)=\sum_{x\in \mathcal{X}} p(x)l(x)$成为$C$的平均码长,则有$L(C)\geq H_D(X)$,且等号成立的充要条件是$D^{-l_i} = p_i$
\end{thm}
\begin{remark}
使得$L(C)$最小的码称为最优码。
\end{remark}
\begin{proof}
由定理\ref{thm:Kraft},设$r=\sum_{x\in \mathcal{X}} D^{-l(x)}\leq 1$
则$q(x) = \frac{D^{-l(x)}}{r}$ 是一个概率分布。
\begin{align*}
    L(C) - H_D(X) & = \sum_{x \in \mathcal{X}} p(x)[l(x)+\log_D(p(x))] \\
                  & = \sum_{x \in \mathcal{X}} p(x)[\log_D(p(x))-\log_D D^{-l(x)}] \\
                  & = \sum_{x \in \mathcal{X}} p(x)\left[\log_D(p(x))-\log_D \frac{D^{-l(x)}}{r}\right] -\log_D r \\
                  & = D(p || q) -\log_D r \\
                  &\geq 0
\end{align*}
上式等号成立当且仅当$r=1$且$p=q$,即$D^{-l_i} = p_i$,从而要求$-\log_D p_i$ 为整数。
\end{proof}
\begin{thm}
    设$C$为最优码,则 $L(C) < H_D(X) + 1 $
\end{thm}
\begin{proof}
只需构造一种编码方式$C'$使得$L(C') < H_D(X)+1$。
为此,
设$X\sim p(x),p(x_i)=p_i,$ 取 $ l_i = \ceil{-\log_D p_i} $
因为 $ l_i \geq \log_D p_i \Rightarrow D^{-l_i}\leq p_i \Rightarrow \sum_i D^{-l_i} \leq 1 $
所以存在一种即时码$C'$ 码长分别为 $ l_i $。
另一方面 $l_i < -\log_D p_i +1 \Rightarrow$
$$
L(C) = \sum_i p_i l_i < \sum_i p_i(-\log_D p_i + 1) = H_D(X)+1
$$
\end{proof}
\item Huffmann 码

\begin{eg}
见\texttt{3.tex} 第7题。
\end{eg}
\begin{thm}
设$C^*$为 Huffman 码,$C$为任意编码,则 $L(C^*)\leq L(C)$
\end{thm}
\begin{proof}
以二元编码($D=2$)为例:
对$|\mathcal{X}|$使用归纳法,当$|\mathcal{X}|=2$时显然成立。假设结论对任意给定的$|\mathcal{X}|\leq m-1$成立。
考虑$\mathcal{X}=\{x_1,\dots,x_m\}$,$p_1\geq p_2\geq\dots\geq p_{m-1}\geq p_m$。
$C$是$\mathcal{X}$上的任意即时码,不妨设$C$满足$l_1 \leq l_2 \leq \dots \leq l_{m-1}\leq l_m$。
否则通过交换码字由排序不等式可以得到$l_1 \leq l_2 \leq \dots \leq l_{m-1}\leq l_m$的编码$C'$使得$L(C')\leq L(C)$。
因此,$C$对应的概率最小的两个码字是最长的两个码字,进一步设它们有相同的长度,否则由即时码的性质将最长码字的末位去掉可以得到平均码长$L(C)$更小
的编码方案。因此$l_{m-1}=l_m$。

考虑缩减信源$\mathcal{X}'$,其中$p'_i=p_i,i=1,\dots,m-2,p'_{m-1}=p_{m-1}+p_m$,设$C_1^*$为$\mathcal{X}'$的Huffman编码,
根据归纳假设对于任意$\mathcal{X}'$上的即时码$C_1$,有$L(C_1^*)\leq L(C_1)$。

另一方面:$L(C)=L(C'_1)+p_{m-1}+p_m$
其中$C'_1$是$\mathcal{X}'$上的编码方法,其由$C$诱导出,诱导规则为,对于前$m-2$个字元码元不变,码长仍为$l_1,\dots,l_{m-2}$,对于第$m-1$个字元,由于$C$是即时码,
将原来$x_{m-1}$或$x_{m}$的码字最后一位去掉可作为第$m-1$个字元的码字,码长为$l_{m-1}-1$。

$\Rightarrow L(C) \geq L(C_1^*)+p_{m-1}+p_m$,不等式右端为常数(给定$\mathcal{X}$)。
\begin{align*}
L(C_1^*)+p_{m-1}+p_m & = \sum_{i=1}^{m-2} ( p'_i l^*_i )+ l^*_{m-1}p'_{m-1}+p_{m-1}+p_m \\
                      & = \sum_{i=1}^{m-2} ( p_i l^*_i )+ (l^*_{m-1}+1)p_{m-1}+(l^*_{m-1}+1)p_m \\
                      & = \sum_{i=1}^{m} p_i l^*_i 
\end{align*}
由 Huffmann编码的构造过程可知,上式等于 $L(C^*)$, 其中$C^*$是$\mathcal{X}$的 Huffman编码 $\Rightarrow L(C)\geq L(C^*)$。
根据归纳法可知对任意有限字母表,均有$L(C)\geq L(C^*)$。
\end{proof}
\item Shannon-Fano-Elias 码,
符号约定,$\bar{F}(x)$ 为修正的累积分布函数, $\bar{F}(x) = \sum_{a<x} p(a) + {1 \over 2} p(x) $
给定 字母表有5个字母,概率分别为 $ 0.25, 0.25, 0.2, 0.15, 0.15 $
则编码如下表所示:
\begin{table}[!ht]
\centering
\begin{tabular}{ccccccc}
\hline
$x$ & $ p(x) $ & $\bar{F}(x)$ & $l(x) = \ceil{\log{1 \over p(x)}}+1$ & $\bar{F}(x) l(x)$位二进制表示 & 码字 \\
\hline
1 & 0.25 & 0.125 & 3 & 0.001 & 001\\
2 & 0.25 & 0.375 & 3 & 0.011 & 011\\ 
3 & 0.2 & 0.6 & 4 & 0.1001 & 1001\\
4 & 0.15 & 0.775 & 4 & 0.1100 & 1100\\
5 & 0.15 & 0.925 & 4 & 0.1110 & 1110\\
\hline
\end{tabular}
\end{table}
上述编码的平均码长为3.5。
\end{enumerate}
\end{document}

