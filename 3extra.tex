\documentclass{article}
\usepackage{ctex}
\usepackage{amsmath,amsthm,amssymb}
\usepackage{enumitem}
\usepackage{mathtools}
\DeclarePairedDelimiter\ceil{\lceil}{\rceil}

\def\E{\mathbb{E}}
\def\Var{\textrm{Var}}
\begin{document}
\begin{enumerate}
\item
\begin{enumerate}[label=(\alph*)] 
\item 由大数定律
$$
\frac{1}{n}\log q(X_1,\dots,X_n) = \frac{1}{n}\sum_{i=1}^n \log q(X_i) \xrightarrow{P}  \sum_{j=1}^m p(x_j)\log q(x_j)
$$
\item 
$$
\frac{1}{n}\log \frac{q(X_1,\dots,X_n)}{p(X_1,\dots,X_n)} \xrightarrow{P} -\sum_{j=1}^m p(x_j)\log\frac{p(x_j)}{q(x_j)} = -D(p||q)
$$
\end{enumerate}

\item
\begin{enumerate}[label=(\alph*)] 
\item $A^n$是典型集,由典型集的性质 $|A^n| \leq 2^{n(H+\epsilon)}$ 及 $|A^n \cap B^n | \leq 2^{n(H+\epsilon)}$ 可得。
\item 由大数定律,当$n$充分大时有
\begin{align*}
\Pr(A^n) & =  \Pr\{|-\frac{1}{n} \log p(X_1,\dots,X_n)-H(X)|  <\epsilon \}  & \geq \frac{3}{4}\\
\Pr(B^n) & =  \Pr\{|-\frac{1}{n}\sum_{i=1}^n x_i - \mu |\leq \epsilon \} & \geq \frac{3}{4}
\end{align*}
$\Rightarrow \Pr(A^n \cap B^n)=1-\Pr(\bar{A}^n)-\Pr(\bar{B}^n)\geq \frac{1}{2}$
\begin{align*}
|A^n \cap B^n| & = \sum_{x\in A^n \cap B^n} 1 \\
                & \geq \sum_{x\in A^n \cap B^n} p(x^n)2^{n(H-\epsilon)} \\
                & = \Pr(A^n \cap B^n)2^{n(H-\epsilon)} \\
                & \geq (1/2)2^{n(H-\epsilon)} 
\end{align*}

\end{enumerate}
\item
\begin{enumerate}[label=(\alph*)] 
\item $\log (1+\binom{100}{1}+\binom{100}{2}+\binom{100}{3}) \approx 17.3$,所以至少需要18位长的码字。
\item $ p = 1-\sum_{i=1}^3 \binom{100}{i}p(0)^{101-i}p(1)^{i-1}=1.6‰$
\item 设$Y=\sum_{i=1}^{100} X_i \Rightarrow \E[Y]=0.5,\Var[Y]=0.4975$
由 Chebyshev 不等式 
$$
\Pr[|Y-\E[Y]|\geq a]\leq \frac{1}{a^2}\Var[Y]
$$
取$a=3.5$ and consider also that $Y$ take discrete values $1,2,3,4,\dots$. Therefore
$$
\Pr[Y\geq 4] \leq 4\%
$$
which is much larger than the accurate result in (b).
\end{enumerate}
\item
We expand $p = 0.p_1p_2\dots$ as a binary number. Let $U = 0.Z_1Z_2\dots$, the sequence $Z$ treated as a binary number.
It is well known that $U$ is uniformly distributed on $[0,1)$. Thus, we generate $X = 1$ if $U < p$ and $0$ otherwise.

The procedure for generated $X$ would therefore examine $Z_1,Z_2\dots$ and compare with
$p_1, p_2\dots$, and generate $a_1$ at the first time one of the $Z_i$'s is less than the corresponding $p_i$( which means $Z_i=0,p_i=1$, and its probability is $\frac{1}{4}$) and generate a $0$ the first time one of the $Z_i$'s is greater than the corresponding $p_i$'s($\Pr(Z_i=1,p_i=0)=\frac{1}{4}$).
Thus the probability that $X$ is generated after seeing the first bit of Z is the
probability that $Z_1 \neq p 1$, i.e., with probability $\frac{1}{2}$.
Similarly, $X$ is generated after 2 bits of $Z$ if $Z_1 = p_1$ and $Z_2 \neq p_2$, which occurs with probability $\frac{1}{4}$. Thus
\begin{align*}
\E[N] & = 1\cdot \frac{1}{2}+ 2\cdot \frac{1}{4} + 3\cdot \frac{1}{8} + \dots\\
      & = 2
\end{align*}
\item 
\begin{enumerate}[label=(\alph*)] 
\item 
$C1$ is nonsingular, since $00$ can be decoded as $0$ and $0$ or $00$;

$C2$ is nonsingular, since $010$ can be decoded as $0$ and $10$ or $010$;

$C3$ is instantaneous since no code is the suffix of the other codes.

$C4$ is nonsingular, since $1010$ can be decoded as $10,10$ or $101,0$.
\item 
Since code $C$ is nonsingular, $C(x_i)\neq C(x_j) \iff x_i \neq x_j \Rightarrow$
H(C(X))=H(X).

Since code $C$ is nonuniquely decodable code, the state of $C(X^n)$ is less than $X^n$, and by the simple inequality

$
-(p_1+p_2)\log(p_1+p_2) < -p_1\log p_1 - p_2\log p_2 \Rightarrow H(C(X^n))<H(X^n)
$

\end{enumerate}
\item
\begin{enumerate}[label=(\alph*)] 
\item 
Since $-\log q(x)\leq l(x)< -\log q(x)+1 \Rightarrow$
$$
-\sum_{x\in\mathcal{X}}p(x)\log q(x) \leq l(x) < -\sum_{x\in\mathcal{X}}p(x)\log q(x)+1  \Rightarrow
$$
$
H(p)+D(p||q) \leq \E_p [l(X)] < H(p)+D(p||q)+1
$
\end{enumerate}
\item
\begin{enumerate}[label=(\alph*)] 
\item consider $\hat{W}_i$, which can be regarded as probability distribution for $m$ random variables. Then the merge process can be modeled as building Huffman
tree. If we merge $W_i$ and $W_j$, it means that the code sequence length for $x_i$ and $x_j$ added by one. Therefore $\frac{V}{W}$ is the average code length of Huffman encoding. For any other encoding by using binary tree, it is also a scheme of instantaneous codes and by the property of Huffman encoding, we have
$\frac{V}{W}\leq \frac{V'}{W}$, where $V'$ is achieved by arbitrary sequences of pairwise merges. And we can get that $V$ is the minimum.
\item By the property of Huffman encoding, $H(\hat{W}) \leq \frac{V}{W} \leq H(\hat{W})+1 \Rightarrow WH(\hat{W})\leq V \leq WH(\hat{W})+W$
\end{enumerate}
\item
\begin{enumerate}[label=(\alph*)] 
\item
The first thing to recognize in this problem is that the player cannot cover more
than 63 ($1+2+4+8+16+32$) values of $X$ with 6 questions. This can be easily seen by induction.
With one question, there is only one value of $X$ that can be covered. With two
questions, there is one value of $X$ that can be covered with the first question,
and depending on the answer to the first question, there are two possible values
of $X$ that can be asked in the next question. By extending this argument, we see
that we can ask at more 63 different questions of the form "Is $X = i$ ?" with 6
questions. (The fact that we have narrowed the range at the end is irrelevant, if
we have not isolated the value of $X$.)

This observation is important since it is not comman sense. Consider the special case when $p_i=\frac{1}{100}$.
People can really think half-divide is the only optimal solution, however, for any 63 numbers out of $\{1,\dots,100\}$,
we can use half-divide (from median) and the probability of winning all equals $63\%$. Therefore, there are 
at least $\binom{100}{63}$ optimal procedures. For general case, we can pick out the most favorable $63$ outcomes, that is, the largest 63 numbers of $p(i)$. Then proceed as usual to get the maximal expected winnings as 
$\sum_{k=1}^{63}p(j_k)v(j_k)$, where $p(j_1)v(j_1)\geq \dots \geq p(j_{100})v(j_{100})$ and $j_1,\dots,j_{100}$ is a permutation of $1,\dots,100$.

\item 
In this case, we assume we can ask question like "Is $X=2.5$ ?", that is,decimal number is allowed.
Since each number must be guessed before terminiting. The "Yes" or "No"(high or low) sequence produced by the given strategy can be treated as binary encoding of the number from 1 to 100. Then the number of questions is equivalent to the length of code. To maximum $\sum_{x=1}^{100} p(x)(v(x)-l(x))$ for given $p(x),v(x)$, we should 
minimize $\sum_{x=1}^{100} p(x)l(x)$. And Huffman encoding can be used to accomplish this. 

Our strategy follows the Huffman tree. We label each node in the tree with a number. Each leaf node of Huffman tree corresponds to a number from 1 to 100 and non-leaf node are stuffed by decimal numbers (not integer) such that it is larger than the largest in the left sub-tree and smaller than the smallest in the right-tree. Then we start from the root node (with value $a$ e.g.), we ask "Is $X=a$ ?". Is $a$ is not integer (leaf node), then the answer can only be high (follow right sub-tree) or low (follow left sub-tree). Our construction gurantees that when we come to leaf node, we get the right answer.

The upper bound of the expected return is
$\sum_{x=1}^{100} p(x)v(x)-H(X)$
\item 
We solve a problem of optimization.
\begin{align}
\min & \,\sum_{x=1}^{100} p(x)v(x)-H(X) \\
s.t. & \,\sum_{x=1}^{100} p(x)=1
\end{align}
$\Rightarrow p(x) = \frac{\exp(-v(x))}{\sum_{x=1}^{100} \exp(-v(x))}$
and the expected return of the user is $-\log(\sum_{x=1}^{100} \exp(-v(x)))$
\end{enumerate}

\item
\begin{enumerate}[label=(\alph*)] 

\item By the principle of Rearrangement Inequality, taste the bottle of milk with larger probability of souring first. Then we can get the minimum expected
number of tasting as :$[p_1,p_2,\dots,p_6]\cdot[1,2,3,4,5,5]=2.89$
\item See (a) for detail.
\item The number of tasting is equivalent to code length of encoding $X\sim (p_1,p_2,\dots,p_6)$. We should use Huffman encoding to achieve the minimum.
The code lengths are $[2,2,3,3,3,3]$ respectively. Therefore, the minimum is $[p_1,p_2,\dots,p_6]\cdot[1,2,3,4,5,5]=2.54$ if mixing is allowed.
\end{enumerate}
\item
\begin{enumerate}[label=(\alph*)] 
\item If $l_i$ is unconstrained, then to minimize $C$ the constraint inequality can be achieved by equality $\sum 2^{-l_i}=1$. We use Lagrange multiplier to 
calculate the optimal $l_i$ for such a problem by differentiate on $\sum_{i=1}^m (p_i c_i l_i -\lambda 2^{-l_i})-1 \Rightarrow 2^{-l_i} = \frac{p_i c_i}{\lambda\ln 2 } \Rightarrow $
$$
l^*_i = -\log \frac{p_i c_i}{\sum_{i=1}^m p_i c_i} \Rightarrow C^* = u\log u - \sum_{i=1}^m p_i c_i \log (p_i c_i),\text{where } u=\sum_{i=1}^m p_i c_i
$$
\item Combine the smallest two $p_i c_i$ and by the same proof technique we can get the smallest cost code scheme.
\item Since $C_{\textrm{Huffman}}$ is obtained by further imposing integer constrains on $l_i$, we have $C^* \leq C_{\textrm{Huffman}}$.
Next, since $l_i^*$ obtained in (a) satisfies $\sum 2^{-l^*_i} = 1$, let $\tilde{l}_i=\ceil{l^*_i}$. Then $\tilde{l}_i\geq l^*_i$, therefore
$\sum 2^{-\tilde{l}_i} \leq 1 $. By Kraft's inequality, there exists a code scheme $C'$ such that its code length is the sequnce $\{\tilde{l}_1,\dots,\tilde{l}_m\}$. On the other hand, we have $\tilde{l}_i < l^*_i+1$. Multiply this inequality with $p_ic_i$ and sum over $i$, we can get
$C' < C^* + \sum_{i=1}^m p_i c_i$. By the optimal property of Huffman encoding, $C_{\textrm{Huffman}}\leq C'$. In conclusion:
$$
C^* \leq C_{\textrm{Huffman}} \leq C' < C^* + \sum_{i=1}^m p_i c_i
$$ 
\end{enumerate}

\end{enumerate}

\end{document}



