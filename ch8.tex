\documentclass{article}
\usepackage{ctex}
\usepackage{bm}
\usepackage{enumitem}
\usepackage{amsmath,amsthm,amsfonts,amssymb}

\def\P{\textbf{P}}
\def\E{\mathbb{E}}
\usepackage{mathtools}
\DeclarePairedDelimiter\ceil{\lceil}{\rceil}
\DeclarePairedDelimiter\abs{\lvert}{\rvert}
\def\X{\mathcal{X}}
\def\Y{\mathcal{Y}}
\def\W{\mathcal{W}}
\usepackage{tikz}
\begin{document}
\section{多变量联合典型序列}
随机向量 $(X_1, \dots, X_k) $ 的$\epsilon $ 典型 且长度为
$n$ 的序列 $ (x_1, \dots, x_k) $ 所构成的集合 $ A_{\epsilon}^{(n)} $ 定义为
$$
A_{\epsilon}^{(n)} = \{ (x_1, \dots, x_k): \abs{ -{ 1 \over n} \log p(\bm{s}) - H(S)} < \epsilon,
\forall S \subset \{X_1, \dots, X_k\}
\}
$$
其中 $ p(\bm{s}) = \prod_{i=1}^n \Pr\{S_i = s_i\} $
并记$ A_{\epsilon}^{(n)}(S) $ 为 $A_{\epsilon}^{(n)}$ 限制在 $S$ 上的典型集。

性质:
\begin{enumerate}
\item 当$ n$ 充分大时,$ \abs{A_{\epsilon}^{(n)} } \geq 2^{-n(H(S)-2\epsilon)} $
\begin{proof}
当$ n$ 充分大时,有 $\Pr\{A_{\epsilon}^{(n)}(S)\} \geq 1 - \epsilon \geq 2^{-n \epsilon}$
\begin{align}
2^{-n \epsilon} & \leq \sum_{\bm{s} \in A_{\epsilon}^{(n)}(S)} p(\bm{s})  \\
& \leq \sum_{\bm{s} \in A_{\epsilon}^{(n)}(S)} 2^{- n (H(S)-\epsilon)} \\
& \leq \abs{A_{\epsilon}^{(n)}(S)} 2^{- n (H(S)-\epsilon)}
\end{align}
所以 $\abs{A_{\epsilon}^{(n)}(S)} \geq 2^{n(H(S) - 2\epsilon)}$
\end{proof}
\item 若 $ S_1, S_2 \subset \{ X^{(1)}, X^{(2)}, \dots, X^{(k)} \} $。 如果 $(\bm{s}_1, \bm{s}_2)
\in A_{\epsilon}^{(n)}(S_1,S_2)$,则有:
\begin{equation}
\label{eq: s1_c_s2} 2^{-n(H(S_1 | S_2) + 2\epsilon)} \leq p(\bm{s}_1 | \bm{s}_2 ) \leq 2^{-n(H(S_1 | S_2) - 2\epsilon)}
\end{equation}
\begin{proof}
\begin{align}
2^{-n(H(S_1, S_2) + \epsilon)} & \leq p(\bm{s}_1, \bm{s}_2) \leq 2^{-n(H(S_1, S_2) - \epsilon)} \\
2^{-n(H(S_2) + \epsilon)} & \leq p(\bm{s}_2) \leq 2^{-n(H(S_2) - \epsilon)} 
\end{align}
利用 $ p(\bm{s}_1 | \bm{s}_2) = \frac{p(\bm{s}_1, \bm{s}_2)}{p(\bm{s}_2)} $
 和 $ H(S_1 | S_2) = H(S_1,S_2) - H(S_2)$ 即可得证。
\end{proof}
\item 给定 $\bm{s}_2 \in A_{\epsilon}^{(n)}(S_2)$,记 $A_{\epsilon}^{(n)}(S_1 | \bm{s}_2)$ 为与给定
序列 $\bm{s}_2 $  构成联合典型 的所有序列的集合, 则
\begin{align}
\label{eq: s1s2Upper} \abs{A_{\epsilon}^{(n)}(S_1 | \bm{s}_2 ) }& \leq 2^{n ( H(S_1 | S_2) + 2\epsilon)} \\
\label{eq: s1s2Lower}(1 - \epsilon)  2^{n ( H(S_1 | S_2) - 2\epsilon)} & \leq
 \sum_{\bm{s_2}} p(\bm{s}_2)\abs{  A_{\epsilon}^{(n)}(S_1 | \bm{s}_2 ) }
\end{align}
\begin{proof}
根据~\eqref{eq: s1_c_s2} 可得
\begin{align*}
1 & \geq \sum_{\bm{s}_1 \in A_{\epsilon}^{(n)}(S_1 | \bm{s}_2 ) } p(\bm{s}_1 | \bm{s}_2 ) \\
& \geq 2^{-n(H(S_1 | S_2) + 2\epsilon)} \abs{ A_{\epsilon}^{(n)}(S_1 | \bm{s}_2 )}
\end{align*}
另一方面,由 $ 1 - \epsilon \leq \Pr\{  A_{\epsilon}^{(n)}(S_1, S_2) \} $ 可得
\begin{align*}
1 - \epsilon & \leq \sum_{\bm{s}_2} p(\bm{s}_2) \sum_{\bm{s}_1 \in A_{\epsilon}^{(n)}(S_1 | \bm{s}_2 ) } p(\bm{s}_1 | \bm{s}_2 ) \\
& \leq  \sum_{\bm{s}_2} p(\bm{s}_2) 2^{-n(H(S_1 | S_2) - 2\epsilon)} \abs{A_{\epsilon}^{(n)}(S_1 | \bm{s}_2 )}
\end{align*}      
\end{proof}
\end{enumerate}
\section{多接入信道(MAC)}
$m $ 个发送器, 1 个接收器。以  $m = 2 $ 为例,
信道构成 为 $ \{ \mathcal{X}_1, \mathcal{X}_2, p(y | x_1, x_2), \mathcal{Y} \} $。

输入为 两个消息集 $ \mathcal{W}_i = \{ 1, 2, \dots, 2^{nR_i} \}, i = 1, 2 $。

编码函数: $ X_i: \mathcal{W}_i  \to \X^n, i = 1, 2 $

译码函数: $ g: \Y \to \W_1 \times \W_2 $。

多接入信道编码为 $ ((2^{n R_1}, 2^{ n R_2}), n ) $ 码。
\begin{itemize}
\item 独立的二元对称信道, 容量区域为 $ \left\{(R_1, R_2) \Big\vert \begin{array}{c} 0 \leq R_1 \leq 1-H(p_1) \\
0 \leq R_2\leq 1 - H(p_2) \end{array} \right\}$, 其中 $p_1, p_2 $ 分别是两个BSC 的错误概率。
\item 二元乘法信道, $ Y = X_1 X_2 $, 容量区域为 
$ \left\{(R_1, R_2) | R_1\geq 0, R_2\geq 0, R_1 + R_2 \leq 1 \right\}$
\item 二元擦除多接入信道 $Y = X_1 + X_2 $, 对应二元输入有三元输出。当 $Y = 1 $ 时,输入可能是$(0, 1) $ 或 $(1, 0)$。当 $R_1 = 1 $时, 可取 $ X_1 \sim Bern({ 1 \over 2 } ) $ 达到。此时将 $X_1 $
视为噪声,将 $Y$的状态$Y=1$视为二进制擦除信道的中间态,则$X_2 $ 可达码率为 $ { 1 \over 2}$。用 $X_1 = Y - X_2 $ 即可解码 $X_1$。因此,容量区域为 $ \left\{(R_1, R_2)  | 0 \leq R_1 \leq 1, 0 \leq R_2\leq 1, R_1 + R_2 \leq {3 \over 2} \right\}$
\item 高斯多接入信道的信道容量。固定$X_2$, $R_1$ 可达最大值 $ { 1 \over 2} \log(1 + \frac{P_1}{N})$;在 $R_1$ 保持不变的情况下,将 $X_1$ 视为 $X_2 $ 的噪声, 则噪声总功率为 $P_1+N$。
因此 $X_2$ 可达速率为 $ { 1 \over 2} \log (1 + { P_2 \over N + P_1 } ) $ 
对于两输入的高斯多接入信道, 容量区域为 
$\{ (R_1, R_2) | 0 \leq R_1 \leq  { 1 \over 2} \log(1 + \frac{P_1}{N}),  0 \leq R_2 \leq  { 1 \over 2} \log(1 + \frac{P_2}{N}),
R_1 + R_2 \leq { 1 \over 2} \log(1 + \frac{P_1 + P_2}{N})
 \}$
\end{itemize}
\section{广播信道}
1个发送器, $m $ 个接收器。
\begin{description}
\item [面向公众广播] 一个信息编码器,相同译码器独立译码
\item [独立信息广播] 用户信息联合编码, 不同译码器独立译码
\item [带公共信息的独立信息广播] 用户信息和公共信息联合编码,不同译码器独立译码
\end{description}
以 $ m = 2$ 为例,独立信息广播信道的信道构成为 $\{ \X, p(y_1, y_2 | x), \Y_1, \Y_2 \} $, 为 $ ( (2^{ n R_1}, 2^{n R_2}), n ) $
码。
\begin{itemize}
\item 输入消息集为 $ \W_1 \times \W_2 = ( \{1, \dots, 2^{ n R_1 } \} \times \{1, \dots, 2^{ n R_2 } \} ) $
\item 编码函数 $ X: \W_1 \times \W_2 \to \X^n $
\item 译码函数 $ g_i : \Y_i^n \to \W_i, i = 1,2 $
\end{itemize}
对于带公共信息的广播信道, 信道构成与 独立信息广播信道相同,但输入的消息集多出公共部分 
$ \W_0 = \{ 1, \dots, 2^{ n R_0 } \} $,为 $ ( (2^{ n R_0}, 2^{ n R_1}, 2^{n R_2}), n ) $
码。
\begin{itemize}
\item 正交广播信道
\item 独立二元对称广播信道
\item 高斯广播信道
\end{itemize}
\end{document}







