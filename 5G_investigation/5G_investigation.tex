\documentclass{ctexart}
    \usepackage{hyperref}
\begin{document}
\begin{table}[!ht]
    \centering
    \begin{tabular}{cp{4cm}p{4cm}}
        \hline
            & 优点 & 缺点 \\
        \hline
        LDPC & 逼近香农限 \newline
               具有译码复杂度低 \newline
               可并行译码 \newline
               译码错误可检测
             & 硬件资源需求大 \newline
               编码复杂造成额外消耗 \\
        \hline
        Turbo & 逼近香农限 \newline
        & 迭代次数多 \newline 译码时延较大 \newline
        受限于5G高速率 \newline 和低时延场景 \\
        \hline
        Polar & 逼近香农限 \newline
        编码译码简单 \newline
        纠错性能好 \newline
        商业化设备成本低 & 在短码上无法实现 \\
        \hline
    \end{tabular}
    \caption{三种编码的优缺点\cite{a}}
\end{table}
投票的轮次,一共两轮,1轮是86是当年10月发生在葡萄牙,一轮是87是11月在美国。
86次会议,联想在数据短码上弃权、控制码投票高通,由于数据短码存在争议,本轮投票未确定数据短码最终方案。
87次会议,联想在数据短码上投票华为,但最终华为以微弱的票数不敌高通,没拿下数据短码。\cite{b}
\begin{thebibliography}{9}
    \bibitem{a} \href{http://www.eefocus.com/communication/373020/r1}{趣科技 | LDPC与Polar码并非有前世恩怨,只是被卷进帮派之战}
    \bibitem{b} \href{https://www.zhihu.com/question/276752269}{如何看待 5G 标准上联想的投票?}
\end{thebibliography}
\end{document}
