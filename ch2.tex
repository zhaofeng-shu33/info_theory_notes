\documentclass{article}
\usepackage{ctex}
\usepackage{bm}
\usepackage{enumitem}
\usepackage{amsmath,amsthm,amsfonts,amssymb}
\usepackage{titleps}
\newtheorem{definition}{定义}
\newtheorem{thm}{定理}
\newtheorem{pro}{命题}
\newtheorem{cor}{推论}
\def\P{\textbf{P}}
\def\E{\mathbb{E}}
% redefine \makeheadrule
\newpagestyle{ch2}{\sethead{}{\thepage}{信息论}
\def\makeheadrule{
\rule[-.3\baselineskip]{\linewidth}{1pt}
\hskip-\linewidth\rule[-.4\baselineskip]{\linewidth}{0.5pt}}
}
\begin{document}
\pagestyle{ch2}
\begin{enumerate}
\item 熵的性质

离散熵的定义为
\begin{definition}
随机变量$X\sim p(x),p(x_i)=p_i,i=1,\dots,n,x_i \in \mathcal{X}=\{x_1,\dots,x_n\}$,
$$
H(p_1,\dots,p_n)\triangleq -\sum_{i=1}^n p_i \log p_i = \E[\log p(X)]
$$
其中$\log$以2为底,熵的单位是比特。
\end{definition}
熵函数的性质:
\begin{itemize}
\item 非负性
\item 是$n$的增函数
\item 可加性 $$H(p_1,p_2,\dots,p_n)=H(\sum_{i=1}^k p_i,p_{k+1},\dots,p_n)+\sum_{i=1}^k p_i H(p'_1,\dots,p'_k)$$
其中$p'_i=p_i/\sum_{i=1}^k p_i$
\item 对称性,若$\sigma$为$1,\dots,n$上的一个置换,则:
$$H(p_1,\dots,p_n)=H(p_{\sigma(1)},\dots,p_{\sigma(n)}) $$
\item 称$h(p)\triangleq H(p,1-p)$为二元熵函数,易证$h(p)$是上凸函数。一般地,设$\textbf{P}=(p_1,\dots,p_n)$,
所有长为$n$的概率向量$\textbf{P}$组成一个凸域$D$。利用$-p\log p$函数的上凸性可以证明$\forall \textbf{P},\textbf{P}'\in D$有
$$H(\lambda \textbf{P}+(1-\lambda)\textbf{P})\geq \lambda H(\textbf{P})+(1-\lambda) H(\textbf{P}')$$
即$H(\textbf{P})$是$\textbf{P}$的上凸函数。
\end{itemize}
\item 联合熵和条件熵
\begin{definition}
一对离散型随机变量$(X,Y)$,联合分布为$p(x,y)$,它们的联合熵为
$$H(X,Y)\triangleq -\sum_{\substack{x\in \mathcal{X} \\ y\in \mathcal{Y}}} p(x,y)\log p(x,y) = -\E[\log p(X,Y)]$$
\end{definition}
\begin{definition}
$Y$对于$X$的条件熵定义为
\begin{align*}
H(Y|X)\triangleq & \E_X[H(Y|X=x)] \\
= & \sum_{x\in \mathcal{X}} p(x)H(Y|X=x)\\
= & \sum_{x\in \mathcal{X}} p(x)[-\sum_{y\in \mathcal Y}p(y|X=x)\log p(y|X=x)] \\
= & -\sum_{x\in \mathcal{X},y\in \mathcal{Y}} p(x,y)\log p(y|X=x)\\
= & -\E_{X,Y}[\log p(Y|X)]
\end{align*}
\end{definition}

关系式:
\begin{itemize}
\item $H(X,Y)=H(X)+H(Y|X)=H(Y)+H(X|Y)$
\begin{proof}
\begin{align*}
H(X,Y)= & -\sum_{x\in \mathcal{X},y\in \mathcal{Y}} p(x,y)\log p(x,y)\\
= & -\sum_{x\in \mathcal{X},y\in \mathcal{Y}} [p(x,y)\log p(y|X=x)+p(x,y)\log p(x)]\\
= & -\sum_{x\in \mathcal{X},y\
in \mathcal{Y}} p(x,y)\log p(y|X=x)-\sum_{x\in \mathcal{X}} p(x)\log p(x)\\
= & H(Y|X)+H(X)
\end{align*}
\end{proof}
\item 记$H(X_1 | X_0)=H(X_1)$,则有
$$
H(X_1,\dots,X_n)=\sum_{i=1}^n H(X_i | X_1,\dots, X_{i-1})
$$
\end{itemize}
\item 相对熵
\begin{definition}
设$p(x),q(x)$是$\mathcal{X}$中字母表相同的两个概率分布,则它们的相对熵定义为:
$$
D(p||q)\triangleq \sum_{x\in \mathcal{X}} p(x)\log\frac{p(x)}{q(x)}
= \E_X[\log\frac{p(X)}{q(X)}]
$$
\end{definition}
\begin{thm}
$D(p||q)\geq 0$,等号成立当且仅当$p=q$
\end{thm}
\begin{proof}
由$-\log p$函数的下凸性质,
\begin{align*}
D(p||q) =& \E_{X}[ -\log\frac{q(X)}{p(X)}],\text{by Jensen's Inequality} \\
\geq &  -\log\left(\E_{X} [\frac{q(X)}{p(X)}]\right)\\
= & 0
\end{align*}
由 Jensen 不等式的取等条件,$\frac{p}{q}$应为常数。
另一种证明是利用不等式 $\log x \leq x - 1$:
$D(p||q) \geq \E_{X}[1 - \frac{q(x)}{p(x)}] = 0$.
\end{proof}
利用相对熵的非负性,我们可以证明
\begin{cor}
设$X$是在字母表$\mathcal{X}$上取值的随机变量,则$H(X)\leq \log|\mathcal{X}|$,等号成立当且仅当$X$是均匀分布。
\end{cor}
\begin{proof}
设$u$是$\mathcal{X}$上的均匀分布,则有$D(X|u)\geq 0 \Rightarrow H(X)\leq \log|\mathcal{X}|$
\end{proof}

相对熵$D(p||q)$的下凸性可以总结为以下三点:
\begin{itemize}
\item $q$固定,由 $t\log t$的下凸性可以得到
$$
D(\lambda p+(1-\lambda)p'||q)\leq \lambda D(p||q)+(1-\lambda)D(p'||q) 
$$
\item $p$固定,由 $-\log t$的下凸性可以得到
$$
D(p||\lambda q+(1-\lambda)q')\leq \lambda D(p||q)+(1-\lambda)D(p||q') 
$$
\item 二元凸性
$$
D(\lambda p+(1-\lambda)p'||\lambda q+(1-\lambda)q')\leq \lambda D(p||q)+(1-\lambda)D(p'||q') 
$$
\begin{proof}
由对数和不等式
\begin{equation}
\sum_{i=1}^n a_i\log \frac{a_i}{b_i} \geq (\sum_{i=1}^n a_i)\log \frac{\sum_{i=1}^n a_i}{\sum_{i=1}^n b_i}
\end{equation}
\begin{align*}
D(\lambda p+(1-\lambda)p'||\lambda q+(1-\lambda)q') = & \sum_{x\in \mathcal{X}} (\lambda p(x)+(1-\lambda)p'(x))\log \frac{\lambda p(x)+(1-\lambda)p'(x)}{\lambda q(x)+(1-\lambda)q'(x)}\\
\leq & \sum_{x\in \mathcal{X}} [\lambda p(x)\log \frac{p(x)}{q(x)}+(1-\lambda) p'(x)\log \frac{p'(x)}{q'(x)}]\\
=& \lambda D(p||q) + (1-\lambda)D(p'||q')
\end{align*}
\end{proof}
\end{itemize}
\item 互信息
\begin{definition}
设$(X,Y)\sim p(x,y),x\in \mathcal{X},y\in \mathcal{Y}$, $X,Y$的互信息
定义为
$$
I(X;Y)\triangleq \sum_{\substack{x\in \mathcal{X} \\ y\in \mathcal{Y}}} p(x,y)\log \frac{p(x,y)}{p(x)p(y)} 
$$
\end{definition}
互信息量有如下的性质:
\begin{itemize}
\item $I(X;Y)=D(p(x,y)||p(x)p(y))\Rightarrow I(X,Y)\geq 0$且 
$I(X;Y)=0 \iff p(x,y)=p(x)p(y)$
\item $I(X;Y)=H(X)-H(X|Y)=H(Y)-H(Y|X)=H(X)+H(Y)-H(X,Y)$
\item $I(X;Y|Z)=0 \iff p(x,y|z)=p(x|z)p(y|z)$($X,Y$关于$Z$条件独立)$\iff  X\rightarrow Z \rightarrow Y$ 构成马氏链
\item 互信息的链式法则
$$ I(X_1,\dots,X_n;Y)=\sum_{i=1}^n I(X_i;Y|X_1,\dots,X_{i-1})
$$
\begin{proof}
\begin{align*}
I(X_1,\dots,X_n;Y) = & H(X_1,\dots,X_n)- H (X_1,\dots,X_n | Y )\\
= & \sum_{i=1}^n H(X_i|X_1,\dots,X_{i-1})-\sum_{i=1}^n H(X_i|X_1,\dots, X_{i-1},Y)\\
= & \sum_{i=1}^n I(X_i;Y|X_1,\dots,X_{i-1})
\end{align*}
\end{proof}
\end{itemize}

互信息$I(X;Y)$可以看成是$p(x),p(y|x)$的泛函数:
$$
I(X;Y)=\sum_{\substack{x\in \mathcal{X}\\ y\in \mathcal{Y}}} p(x)p(y|x) \log \frac{p(y|x)}{\sum_{x\in \mathcal{X}} p(x)p(y|x)}
$$
关于互信息的凸性可以总结为以下两点:
\begin{itemize}
\item $I(X;Y)$关于$p(x)$上凸
\item $I(X;Y)$关于$p(y|x)$下凸。
\end{itemize}
数据处理不等式:设 $ P_Y = P_{Y|X}\circ P_X, Q_Y = P_{Y|X} \circ Q_X $
则 $D(P_X || Q_X) \geq D(P_Y || Q_Y) $。
\begin{proof}
$\E_{XY}[\log \frac{P_{XY}}{Q_{XY}}] = 
\E_{XY}[\log \frac{P_{Y|X}}{Q_{Y|X}} + \log \frac{P_X}{Q_X}]$
因为 $Q_{Y|X} = P_{Y|X} \Rightarrow \E_{XY}[\log \frac{P_{XY}}{Q_{XY}}]  = D(P_X || Q_X)$
又
\begin{align*}
\E_{XY}[\log \frac{P_{XY}}{Q_{XY}}] = &
\E_{XY}[\log \frac{P_{X|Y}}{Q_{X|Y}} + \log \frac{P_Y}{Q_Y}] \\
& = \E_{Y}[D(P_{X|Y} || Q_{X|Y})] + D(P_Y || Q_Y) \\
& \geq D(P_Y || Q_Y)
\end{align*}
所以有$D(P_X || Q_X) \geq D(P_Y || Q_Y) $。
\end{proof}
\item 微分熵
设$X$是连续型随机变量,$X$的微分熵定义为:
\begin{equation}\label{eq:differential_entropy}
h(X)\triangleq -\int_{\mathbb{R}} p(x)\log p(x)dx
\end{equation}
$X^{(\Delta)}$ 是$X$按区间长度为$\Delta$离散的结果。
则 $H(X^{(\Delta)})+\log \Delta \to h(X)$


微分熵可正可负

常见分布的微分熵
\begin{pro}
\begin{enumerate}[label=(\alph*)]
\item 区间$(0, a)$ 上的均匀分布的微分熵为 $ \log a $, 可正可负。
\item 方差为$\sigma^2$的高斯分布微分熵的大小为$\frac{1}{2}\log (2\pi e \sigma^2)$,协方差矩阵为$\Sigma$的联合高斯分布微分熵大小为${1\over 2} \log(\det (2\pi e \Sigma ))$
\item 均值为$\frac{1}{\lambda}$的指数分布的微分熵为 $\log e-\log \lambda$
\end{enumerate}
\end{pro}
\begin{proof}
\begin{enumerate}[label=(\alph*)]
\item
不妨设高斯分布$X$的均值为0,概率密度函数为$p(x)$则
\begin{align*}
h(X) = & -\int_{\mathbb{R}} p(x)\log\left(\frac{1}{\sqrt{2\pi \sigma^2}}\exp(-\frac{x^2}{2\sigma^2})\right)dx\\
=& \frac{1}{2}\log(2\pi \sigma^2)+\frac{\log e}{2\sigma^2} \int_{\mathbb{R}} x^2p(x)dx\\
=& \frac{1}{2}\log(2\pi e\sigma^2)
\end{align*}
\item 
$$
h(X)=-\int_{0}^{\infty} \lambda e^{-\lambda x}(\log \lambda -\lambda x \log e)dx=\log e-\log \lambda
$$
\end{enumerate}
\end{proof}

\item 相对熵和互信息(连续情形)
\begin{definition}
若$X\sim p(x),Y\sim q(y)$连续,则$X$和$Y$的相对熵为
\begin{equation}
D(p||q)\triangleq \int_{\mathbb{R}} p(x)\log \frac{p(x)}{q(x)}dx
\end{equation}
$X$和$Y$的互信息为:$I(X;Y)=D(p(x,y)||p(x)q(y))$
\end{definition}

常见的连续型概率分布可以看成是某种条件下的最大熵分布:
\begin{thm}
对概率密度族$\mathcal{P}$,若存在$p_0(x)\in \mathcal{P}$,使得
$$\forall p(x)\in \mathcal{P},-\int_{\mathbb{R}} p(x)\log p_0(x)dx =h_0
$$
 是一个与$p(x)$无关的常数,则$p_0(x)$为最大熵分布,
$h_0$为最大熵。
\end{thm}
\begin{proof}
由相对熵的非负性得:
$$
-\int_{\mathbb{R}} p(x)\log p(x)dx \leq -\int_{\mathbb{R}} p(x)\log p_0(x)dx=h_0
$$
等号成立当且仅当$p(x)$与$p_0(x)$几乎处处相等。
\end{proof}
\end{enumerate}
\end{document}

