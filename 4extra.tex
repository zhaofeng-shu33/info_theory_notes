\documentclass{article}
\usepackage{ctex}
\usepackage{amsmath,amsthm,amssymb}
\usepackage{enumitem}
\usepackage{mathtools}
\DeclarePairedDelimiter\ceil{\lceil}{\rceil}

\def\E{\mathbb{E}}
\def\Var{\textrm{Var}}
\usepackage{hyperref}
\begin{document}
\begin{enumerate}
\item
\begin{enumerate}[label=(\alph*)] 
\item $I(X;Y) = H(Y) - H(Y|X) = H(Y) - \log 3 \leq \log 11 - \log 3 = \log { 11 \over 3 } $
所以 $ C = \log { 11 \over 3} $
\item 当 $X$ 等概时达到信道容量。 $ p(x) = { 1 \over 11} , x \in \{ 0, 1, 2, \dots, 10 \}$
\end{enumerate}

\item
\begin{enumerate}[label=(\alph*)] 
\item 设 $Q = \begin{bmatrix} 1- p & p \\ p & 1-p \end{bmatrix}$
$X_0$ 到 $X_n$ 的转移概率矩阵为 $Q^n  = { 1 \over 2}
\begin{bmatrix} 1+ (1 -2 p)^n & 1 - (1 - 2p)^n \\
1 - (1 - 2p)^n & 1 + (1 - 2p)^n 
\end{bmatrix} $
所以$n$个二进制对称信道的级联等价于一个 错误概率 为 $ p_e =  { 1 \over 2} (1 - (1 - 2p)^n ) $ 的对称信道。
若 $ p \neq 0, 1 $, 则 $ -1 < 1 - 2 p < 1 $ 。 当 $ n \to \infty $ 时, $ p_e \to { 1 \over 2} $。此时 $ I (X_0;X_n ) = 0 $。 
\end{enumerate}
\item
\begin{enumerate}[label=(\alph*)] 
\item 
该DMC信道的转移概率矩阵为 $ Q = \begin{bmatrix} 1 - \alpha - \epsilon & \epsilon & \alpha \\
\epsilon & 1 - \alpha - \epsilon & \alpha 
\end{bmatrix}$
为一准对称信道,当输入分布等概时达到信道容量。
此时 $ Y $ 的分布为 $ \begin{pmatrix} 0 & 1 & e \\ { 1 - \alpha \over 2} & { 1 - \alpha \over 2} & \alpha \end{pmatrix} $,
$ Y | X = i $ 的分布为 $Q$ 的 第 $ ( i + 1 ) $ 行。于是可以求出
\begin{align*}
 C & = I (X;Y) = H(Y) - H(Y|X) \\
 & = (1 - \alpha - \epsilon ) \log ( 1 - \alpha - \epsilon ) + \epsilon \log \epsilon - (1 - \alpha ) \log \frac{ 1 - \alpha }{ 2}
\end{align*}
\item 当 $ \alpha = 0 $ 时(BSC), $ C = 1 - h(\epsilon)$, 其中 $ h(\epsilon) $ 为二元熵函数。
\item 当 $ \epsilon = 0 $ 时(BEC), $ C = 1 - \alpha $。
\end{enumerate}

\item 
\begin{enumerate}[label=(\alph*)] 
\item 
$ Y = 2 X + Z_1 + Z_2 \Rightarrow h(Y|X) = h(Z_1 + Z_2) $。因为 $\Var[Z_1 + Z_2 ] =  2(1+ \rho) \sigma^2 $,
所以 $ h(Z_1 + Z_2) = { 1 \over 2} \log (2\pi e \Var[Z_1 + Z_2] )$。
又因为 $ \Var[Y] = \Var[2X + Z_1 + Z_2 ] = 4 P + \Var[Z_1 + Z_2 ]  $  为定值, 当 $ Y $ 是高斯分布时 $h (Y) $ 
最大,此时 $ h^*(Y) = { 1 \over 2} \log (2\pi e \Var[Y] )$
$ \Rightarrow C = h^*(Y) - h(Y|X) = {1 \over 2} \log ( 1 + { 2P \over (\rho+1) \sigma^2})$
\item 当 $ \rho = 1 $ 时, $ Z_1 \overset{as}{=} Z_2  \Rightarrow C = { 1 \over 2} \log ( 1 + { P \over \sigma^2})$;
当 $ \rho = 0 $ 时, $ Z_1 $ 与$ Z_2 $ 独立, $ \Rightarrow C = { 1 \over 2} \log ( 1 + { 2P \over \sigma^2})$;
当 $ \rho = -1 $ 时, $ Z_1 \overset{as}{=} -Z_2 \Rightarrow C = h^*(Y) = h^*(2X) = { 1\over 2} \log (8\pi e P) $
\end{enumerate}
\item
总的输入功率的约束为 $ \E [ \sum_{j = 1}^k X_j^2 ]\leq P $
根据$(x)^{+} = \begin{cases} x & x \geq 0 \\ 0 & x< 0 \end{cases}$的定义 可得要证的结论。
\item (针对$DMC$ 信道的信道编码定理证明)由第10题推导得出的不等式 
$ R \leq { 1 \over 1 - P_e^{(n)} } ({ 1 \over n}  + C)$
因为 $ P_e^{(n)} \leq \lambda^{(n)} \to 0 $ 所以 令 $ n \to \infty $ 得到 $ R \leq C$
\item 根据限带加性高斯白噪声信道的香农公式,$R \leq W \log (1 + { E \over \sigma } ) $ 代入数据得 $ E \geq 255\mu W$
\item
根据 \href{https://en.wikipedia.org/wiki/Z-channel_(information_theory)}{wikipedia}, 设输入分布为 $Bern(p) $,
当 $ p = { 2 \over 5} $ 时, 达到信道容量 $ \log { 5 \over 4} $。
\item $ \Pr\{ (X^n, Y^n, Z^n ) \in A_{\epsilon}^{(n)} \} \leq 2^{ -n ( H(X) + H(Y) + H(Z) - H(X, Y, Z) - 4 \epsilon ) } $

$ \Pr\{ (X^n, Y^n, Z^n ) \in A_{\epsilon}^{(n)} \} \geq (1-\epsilon)2^{ -n ( H(X) + H(Y) + H(Z) - H(X, Y, Z) + 4 \epsilon ) } $(当 $n$ 充分大时)
\item 设 $W$ 是均匀分布的 $ J = 2^{nR}  \Rightarrow nR = H(W)$。 由 Fano 不等式 $H ( W | \widehat{W} ) \leq 1 + P^{(n)}_e n R $

所以 
\begin{align*}
nR & = H(W) \\
& = H(W | \widehat{W}) + I(W; \widehat{W}) \\
& \leq 1 + P^{(n)}_e n R + I(X^n ; Y^n) \\
& \leq 1 + P^{(n)}_e n R + nC
\end{align*}
即 $ P_e = {1 \over n}P_e^{(n)} \geq {1 \over \log J}(R - C - {1 \over n}) $
\end{enumerate}

\end{document}



