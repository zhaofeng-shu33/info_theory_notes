\documentclass{article}
\usepackage{ctex}
\usepackage{bm}
\usepackage{enumitem}
\usepackage{amsmath,amsthm,amsfonts,amssymb}

\def\P{\textbf{P}}
\def\E{\mathbb{E}}
\usepackage{mathtools}
\DeclarePairedDelimiter\ceil{\lceil}{\rceil}
\DeclarePairedDelimiter\abs{\lvert}{\rvert}
\def\X{\mathcal{X}}
\def\Y{\mathcal{Y}}
\usepackage{tikz}
\newtheorem{definition}{定义}
\newtheorem{theorem}{定理}
\author{赵丰}
\title{最大熵和最小鉴别原理}
\begin{document}
\maketitle
最大熵原理问题的提法:
某离散随机变量$X$,其概率分布$p(x)$ 未知,已知若干函数在 $p(x)$下的期望:
$ \sum_{x \in \X} p(x) f_m(x) = C_m, m =1, 2, \dots, M$, 求最佳估计 $\hat{p}(x)$。

求解原理是取概率分布的熵为目标函数
$ H(X) = - \sum_{x \in \X} p(x) \log p(x)$, 而
\begin{equation}
\hat{p}(x) = \arg\max_{p(x)} H(X)
\end{equation}

欠定约束下最大熵分布满足如下形式:
\begin{equation}\label{eq:Maximum_Entropy}
\hat{p}(x) = \exp\left[ - \lambda_0 - \sum_{ m = 1 }^M \lambda_m f_m(x) \right]
\end{equation}
其中参数$\lambda_0, \lambda_i (i=1, \dots, M)$ 由 $M+1 $个约束条件确定(包含$\sum_{x \in \X} p(x) = 1$)。

对于连续型随机变量,用积分代替求和,约束条件为

$ \int_S p(x)=1, \int_S p(x) f_m(x) = C_m, m = 1, 2, \dots, M$
则最大熵分布同~\eqref{eq:Maximum_Entropy}式。

最小鉴别信息原理:欠定问题且有先验分布($q(x)$)。

取概率分布的鉴别信息(相对熵)为目标函数:$D(p || q) = \sum_{x \in \X} p(x) \log { p(x) \over q(x) } $
\begin{equation}
\hat{p}(x) = \arg \min_{p(x)} D( p || q)
\end{equation}
用 Lagrange 乘子法可以求出欠定约束下最小鉴别信息分布为满足如下形式:
\begin{equation}\label{eq:Principle_of_Minimum_Discrimination_Information}
\hat{p}(x) = q(x)\exp\left[ \lambda_0 + \sum_{ m = 1 }^M \lambda_m f_m(x) \right]
\end{equation}

\end{document}







